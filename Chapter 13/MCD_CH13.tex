\documentclass[11pt,a4paper,titlepage]{article}
\usepackage[a4paper]{geometry}
\usepackage[utf8]{inputenc}
\usepackage[german]{babel}
\usepackage{lipsum}

\usepackage{amsmath, amssymb, amsfonts, amsthm, fouriernc, mathtools}
\usepackage{microtype} %improves the spacing between words and letters

\usepackage{graphicx}
\graphicspath{{./pics/}}

\usepackage{tabularx}
\usepackage{booktabs}
\usepackage{wasysym}
\usepackage{color, colortbl}


%%%%%%%%%%%%%%%%%%%%%%%%%%%%%%%%%%%%%%%%%%%%%%%%%%
%% COLOR DEFINITIONS
%%%%%%%%%%%%%%%%%%%%%%%%%%%%%%%%%%%%%%%%%%%%%%%%%%
\usepackage[svgnames]{xcolor} % Enabling mixing colors and color's call by 'svgnames'
%%%%%%%%%%%%%%%%%%%%%%%%%%%%%%%%%%%%%%%%%%%%%%%%%%
\definecolor{MyColor1}{rgb}{0.2,0.4,0.6} %mix personal color
\newcommand{\textb}{\color{Black} \usefont{OT1}{lmss}{m}{n}}
\newcommand{\blue}{\color{MyColor1} \usefont{OT1}{lmss}{m}{n}}
\newcommand{\blueb}{\color{MyColor1} \usefont{OT1}{lmss}{b}{n}}
\newcommand{\red}{\color{LightCoral} \usefont{OT1}{lmss}{m}{n}}
\newcommand{\green}{\color{Black} \usefont{OT1}{lmss}{m}{n}}
%%%%%%%%%%%%%%%%%%%%%%%%%%%%%%%%%%%%%%%%%%%%%%%%%%


%%%%%%%%%%%%%%%%%%%%%%%%%%%%%%%%%%%%%%%%%%%%%%%%%%
%% FONTS AND COLORS
%%%%%%%%%%%%%%%%%%%%%%%%%%%%%%%%%%%%%%%%%%%%%%%%%%
%    SECTIONS
%%%%%%%%%%%%%%%%%%%%%%%%%%%%%%%%%%%%%%%%%%%%%%%%%%
\usepackage{titlesec}
\usepackage{sectsty}
%%%%%%%%%%%%%%%%%%%%%%%%
%set section/subsections HEADINGS font and color
\sectionfont{\color{MyColor1}}  % sets colour of sections
\subsectionfont{\color{MyColor1}}  % sets colour of sections

%set section enumerator to arabic number (see footnotes markings alternatives)
\renewcommand\thesection{\arabic{section}} %define sections numbering
\renewcommand\thesubsection{\thesection.\arabic{subsection}} %subsec.num.

%define new section style
\newcommand{\mysection}{
	\titleformat{\section} [runin] {\usefont{OT1}{lmss}{b}{n}\color{MyColor1}} 
	{\thesection} {3pt} {} } 

%%%%%%%%%%%%%%%%%%%%%%%%%%%%%%%%%%%%%%%%%%%%%%%%%%
%		CAPTIONS
%%%%%%%%%%%%%%%%%%%%%%%%%%%%%%%%%%%%%%%%%%%%%%%%%%
\usepackage{caption}
\usepackage{subcaption}
%%%%%%%%%%%%%%%%%%%%%%%%
%\captionsetup[figure]{labelfont={color=Black}}

%%%%%%%%%%%%%%%%%%%%%%%%%%%%%%%%%%%%%%%%%%%%%%%%%%
%		!!!EQUATION (ARRAY) --> USING ALIGN INSTEAD
%%%%%%%%%%%%%%%%%%%%%%%%%%%%%%%%%%%%%%%%%%%%%%%%%%
%using amsmath package to redefine eq. numeration (1.1, 1.2, ...) 
%%%%%%%%%%%%%%%%%%%%%%%%
\renewcommand{\theequation}{\thesection.\arabic{equation}}

%set box background to grey in align environment 
\usepackage{etoolbox}% http://ctan.org/pkg/etoolbox
\makeatletter
\patchcmd{\@Aboxed}{\boxed{#1#2}}{\colorbox{black!15}{$#1#2$}}{}{}%
\patchcmd{\@boxed}{\boxed{#1#2}}{\colorbox{black!15}{$#1#2$}}{}{}%
\makeatother
%%%%%%%%%%%%%%%%%%%%%%%%%%%%%%%%%%%%%%%%%%%%%%%%%%


%%%%%%%%%%%%%%%%%%%%%%%%%%%%%%%%%%%%%%%%%%%%%%%%%%
%% DESIGN CIRCUITS
%%%%%%%%%%%%%%%%%%%%%%%%%%%%%%%%%%%%%%%%%%%%%%%%%%
\usepackage[siunitx, american, smartlabels, cute inductors, europeanvoltages]{circuitikz}
%%%%%%%%%%%%%%%%%%%%%%%%%%%%%%%%%%%%%%%%%%%%%%%%%%


\makeatletter
\let\reftagform@=\tagform@
\def\tagform@#1{\maketag@@@{(\ignorespaces\textcolor{red}{#1}\unskip\@@italiccorr)}}
\renewcommand{\eqref}[1]{\textup{\reftagform@{\ref{#1}}}}
\makeatother

%%%%%%%%%%%%%%%%%%%%%%%%%%%%%%%%%%%%%%%%%%%%%%%%%%
%% PREPARE TITLE
%%%%%%%%%%%%%%%%%%%%%%%%%%%%%%%%%%%%%%%%%%%%%%%%%%
\title{\blue MICROELECTRONIC CIRCUIT DESIGN \\	\blueb Solutions}
\author{Dimitri Haas}
\date{\today}
%%%%%%%%%%%%%%%%%%%%%%%%%%%%%%%%%%%%%%%%%%%%%%%%%%



\begin{document}
\maketitle
\setcounter{section}{12}
\section{Small-Signal Modeling and Linear Amplification}
\setcounter{subsection}{7}
\subsection{Small-Signal Models for Field-Effect Transistors}
\subsubsection*{13.80}
Folgende Werte sind bereits anhand der Aufgabe gegeben:
\begin{equation*}
\begin{aligned}
K_n &= \SI{300}{\micro\ampere\per\square\volt}\\
V_{TN} &= \SI{1}{\volt}\\
\lambda &= \SI{0.02}{\per\volt}
\end{aligned}
\end{equation*}
Ab welchem Drainstrom $I_D$ ist keine Spannungsverstärkung mehr vorhanden, also $\mu_f \leq 1$?\\
\\
Dazu suchen wir in Tabelle 13.3 nach der Formel für $\mu_f$. Diese lautet
\[\mu_f = \frac{1}{\lambda}\sqrt{\frac{2K_n}{I_D}}.\]
Wenn wir nun die Forderung einsetzen und die Gleichung nach $I_D$ umstellen, erhalten wir

\begin{equation*}
\begin{aligned}
 & &\frac{1}{\lambda}\sqrt{\frac{2K_n}{I_D}} &\leq 1 \\
& \Leftrightarrow & \frac{2K_n}{\lambda^2} &\leq I_D \\
& \Leftrightarrow & \frac{2(\SI{300}{\micro\ampere\per\square\volt})}{(\SI{0.02}{\per\volt})^2} &\leq I_D \\
& \Leftrightarrow & \SI{1.5}{\ampere} &\leq I_D
\end{aligned}
\end{equation*}\\
Damit liegt ab einem Drainstrom von $I_D = \SI{1.5}{\ampere}$ keine Spannungsverstärkung mehr vor.

\subsubsection*{13.81}
In dieser Aufgabe soll überprüft werden, ob die Ausdruck
\[\frac{2v_{gs}}{V_{GS}-V_{TN}}\] 
eine geeignete lineare Approximation für Ausdruck
\[\left(1 + \frac{v_{gs}}{V_{GS}-V_{TN}}\right)^2-1\]
darstellt, wenn $v_{gs} = 0.2(V_{GS}-V_{TN})$ gilt.\\
\\
Dazu setzen wir im ersten Schritt in die Ausgangsgleichung das vorgegeben $v_{gs}$ ein und erhalten
\[\left(1 + \frac{0.2(V_{GS}-V_{TN})}{V_{GS}-V_{TN}}\right)^2-1 = (1 + 0.2)^2-1 = 0.44\]
Das gleiche wiederholen wir nun für angenäherten Ausdruck:
\[\frac{(2)(0.2)(V_{GS}-V_{TN})}{V_{GS}-V_{TN}} = 0.4\]
Betrachten wir nun den relativen Fehler
\[f = \frac{0.4 - 0.44}{0.44}\SI{100}{\percent} = \SI{-9.1}{\percent} \approx \SI{-10}{\percent}\]
sehen wir, dass der Fehler bei etwa $\SI{10}{\percent}$ liegt und damit eine gute Näherung darstellt.\\
\\
Führen wir die gleichen Berechnungen für $v_{gs} = 0.4(V_{GS}-V_{TN})$ durch, ergibt sich ein Fehler von
\[f = \frac{0.8 - 0.96}{0.96}\SI{100}{\percent} = \SI{-16.7}{\percent} \approx \SI{-20}{\percent}.\]
Man sieht also, dass das Signal wirklich sehr klein sein muss, um mithilfe der obigen Annäherung ein gutes Ergebnis zu erzielen.

\subsubsection*{13.84}
Gegeben ist ein Common-Source-Amplifier mit folgenden Spezifikationen:

\begin{equation*}
\begin{aligned}
R_\text{out}^{CS} &= \SI{100}{\kilo\ohm}\\
\lambda &= \SI{0.02}{\per\volt}\\
V_{DD} &= \SI{15}{\volt}
\end{aligned}
\end{equation*}
Da ein CS-Amplifier so gestaltet wird, dass etwa \SI{50}{\percent} der Versorgungsspannung über dem Drain-Widerstand abfallen und die Ausgangsspannung weitestgehend der Spannung über dem Drain-Widerstand entspricht, können wir einen Teil des Arbeitspunktes bereits vorzeitig angeben:
\[V_{DS} = \frac{V_{DD}}{2} = \frac{\SI{15}{\volt}}{2} = \SI{7.5}{\volt}\]
Nun gilt es noch einen fehlenden Transistor-Parameter zu bestimmen, welcher vollkommen unabhängig von der äußeren CS-Beschaltung ist:
\begin{equation}\label{eq:r_0} 
r_0 = \frac{\frac{1}{\lambda}+V_{DS}}{I_D}
\end{equation}
Allerdings enthält Gleichung (\ref{eq:r_0}) noch zwei Unbekannte. Wir müssen deshalb noch einen weiteren Zusammenhang ausfindig machen. Fündig wird man, wenn man sich klarmacht, dass der Ausgangswiderstand eine Parallelanordnung der Widerstände $r_0$ und $R_D$ darstellt. Da wir in der Aufgabenstellung seinen Wert übermittelt bekommen haben, können wir schreiben:
\begin{equation}\label{eq:r_out} 
R_\text{out} = R_{\text{out}}^{CS} = \frac{r_0 R_D}{r_0+R_D}
\end{equation}
Setzt man nun Gleichung (\ref{eq:r_0}) in (\ref{eq:r_out}) ein und verwendet die Beziehung $R_D=\frac{V_{DS}}{I_D}$, erhält man:
\begin{equation} \label{eq:r_outAndId}
\begin{aligned}
R_\text{out}^{CS} &= \frac{\frac{\frac{1}{\lambda}+V_{DS}}{I_D}\frac{V_{DS}}{I_D}}{\frac{\frac{1}{\lambda}+V_{DS}}{I_D}+\frac{V_{DS}}{I_D}} \\
&= \frac{1}{I_D}\frac{\frac{1}{\lambda}+V_{DS}}{\frac{1}{\lambda V_{DS}}+2} \\
&\approx \frac{6.63}{I_D}
\end{aligned}
\end{equation}
Da uns $R_\text{out}^{CS}$ bereits in der Aufgabenstellung gegeben wurde, können wir mithilfe von Gleichung (\ref{eq:r_outAndId}) den fehlenden Arbeitspunkt-Parameter bestimmen:
\begin{equation}
I_D = \frac{6.63}{R_\text{out}^{CS}} = \frac{6.63}{\SI{100}{\kilo\ohm}} = \SI{66.3}{\micro\ampere}
\end{equation}
Der Arbeitspunkt lautet somit abschließend $(\SI{7.5}{\volt},\SI{66.3}{\micro\ampere})$. Diese Aufgabe kann man sich mithilfe von Figure 13.18 im Buch auf Seite 793 veranschaulichen. Hier müssen lediglich die BJT- durch MOSFET-Bauteile ersetzt werden.

\subsubsection*{13.85}
In einem CS-Verstärker, ist der Arbeitspunkt gesucht, bei dem wir einen Eingangswiderstand von $R_{\text{in}}^{CS} = \SI{2}{\mega\ohm}$ erhalten. Der darin verbaute Transistor besitzt folgende Parameter:
\begin{equation*}
\begin{aligned}
K_n &= \SI{500}{\micro\ampere\per\square\volt}\\
V_{TN} &= \SI{1}{\volt}\\
\lambda &= \SI{0.02}{\per\volt}
\end{aligned}
\end{equation*}
Gearbeitet wird mit einer Versorgungsspannung von \SI{12}{\volt}.\\
\\
Die obige Aufgabe kann auch ohne Studium des Abschnitts 13.10.5 im Buch gelöst werden: Da der Transistor selbst einen unendlichen hohen Eingangswiderstand hat, bleibt nur noch $R_G$ als Widerstand übrig. Somit ist der Eingangwiderstand nur von den beiden Gate-Widerständen nicht aber vom Arbeitspunkt abhängig. Jeder Arbeitspunkt ermöglicht damit den oben vorgegeben Eingangswiderstand.

\subsection{Summary and Comparison of the Small-Signal Models of the BJT and FET}
\subsubsection*{13.87}
Nur mit einem FET kann ein solch großer Eingangswiderstand realisiert werden. Ein BJT kann bei so einem Arbeitsstrom nur etwa \SI{500}{\ohm} liefern: 
\[r_{\pi}=\frac{\beta_0 V_T}{I_C}=\frac{(100)(\SI{0.025}{\volt})}{\SI{5}{\milli\ampere}}=\SI{500}{\ohm}\]

\subsubsection*{13.88}
Wieder stellt sich die Frage, welcher Transistortyp am besten geeignet ist, um die gegebene Aufgabe erfüllen zu können. Gewünscht ist eine Transkonduktanz von \SI{0.5}{\siemens}. Schauen wir uns dann einfach die Transkonduktanz von beiden Transistoren für die gegebenen Werte an:\\
\begin{enumerate}
	\item \textbf{BJT}: Um eine Transkonduktanz von \SI{0.5}{\siemens} zu erreichen, wird ein Kollektorstrom in Höhe von $I_C=(\SI{0.5}{\siemens})(\SI{0.025}{\volt}) = \SI{12.5}{\milli\ampere}$ benötigt.
	\item \textbf{MOSFET}: Nach gleichem Verfahren ermitteln wir nun Drainstrom $I_D$. Formel umformen ergibt $I_D=\frac{(\SI{0.5}{\siemens})^2}{(2)(\SI{25}{\milli\ampere\per\square\volt})}=\SI{5}{\ampere}$.
\end{enumerate} 
Hier wählt man natürlich den Bipolartransistor, weil dieser einen deutlich geringeren Anspruch an die Stromversorgung stellt.

\subsubsection*{13.89}
Berechnen wir zunächst die intrinsische Verstärkung $\mu_f$ des Bipolartransistoren:
\[ \mu_f = \frac{V_A + V_{CE}}{V_T} = \frac{\SI{60}{\volt} + \SI{10}{\volt}}{\SI{0.025}{\volt}} = 2800 \]
Um die oben berechnete Verstärkung zu erreichen, können wir nun die benötigte Spannung $(V_{GS}-V_{TN})$ berechnen und anschließend mit dieser den Arbeitsstrom bestimmen:
\[ V_{GS}-V_{TN} = \frac{2 \left( \frac{1}{\SI{0.017}{\per\volt}} +\SI{10}{\volt} \right)}{2800} \approx \SI{0.049}{\volt}  \]
Das nun in die Stromgleichung des MOSFET eingesetzt erhalten wir:
\[ I_D = \frac{K_n}{2} \left( V_{GS} - V_{TN} \right)^2(1+\lambda V_{DS}) = \frac{\SI{25}{\milli\ampere\per\square\volt}}{2} (\SI{0.049}{\volt})^2(1+ \SI{0.017}{\per\volt} \SI{10}{\volt}) = \SI{35.1}{\micro\ampere} \]

\subsubsection*{13.90}
Wie gehabt, berechnen wir zunächst die intrinsischen Verstärkungen:
\begin{enumerate}
	\item \textbf{BJT}: $\mu_f \cong \frac{V_A}{V_T} = \frac{\SI{50}{\volt}}{\SI{0.025}{\volt}} = 2000$
	\item \textbf{MOSFET}: $\mu_f = g_mr_0 \cong \frac{2}{\lambda(V_{GS}-V_{TN})} = \frac{2}{\SI{0.02}{\per\volt}\SI{0.5}{\volt}} = 200$
\end{enumerate}
Nun soll für einen Arbeitsstrom von $\SI{200}{\micro\ampere}$ für beide Geräte die Transkonduktanz berechnet werden:
\begin{enumerate}
	\item \textbf{BJT}: $g_m = \frac{I_C}{V_T} = \frac{\SI{200}{\micro\ampere}}{\SI{0.025}{\volt}} = \SI{8}{\milli\siemens}$
	\item \textbf{MOSFET}: $g_m = \frac{2I_D}{V_{GS}-V_{TN}} = \frac{(2)(\SI{200}{\micro\ampere})}{\SI{0.5}{\volt}} = \SI{0.8}{\milli\siemens}$
\end{enumerate}

\subsubsection*{13.91}
Gesucht ist eine Verstärker-Schaltung mit einer Eingangsimpedanz von \SI{50}{\ohm}. Solche kleinen Eingangswiderstände lassen sich sowohl mit einem BJT als auch mit einem MOSFET realisieren. Siehe dazu Tabelle 3.6 in Kapitel 13.11. Durch die Wahl geeigneter Beschaltung kann die geforderte Eingangsimpedanz erfüllt werden:
\begin{enumerate}
	\item \textbf{BJT}: $R_{in} \approx r_{\pi}$ $\rightarrow$ Dieser Eingangswiderstand kann bei einem Arbeitsstrom von $I_C = \frac{\beta_0V_T}{r_{\pi}} = \frac{(100)(\SI{0.025}{\volt})}{\SI{50}{\ohm}} = \SI{50}{\milli\ampere}$ gewährleistet werden. Zu beachten ist, dass das einem relativ hohen Strom entspricht!
	\item \textbf{MOSFET}: $R_{in} = R_G$ $\rightarrow$ Hier kann man einfach einen entsprechenden Widerstand wählen.
\end{enumerate}
Wenn also ein energiesparender Betrieb gewünscht ist, kann in diesem Fall am besten auf den MOSFET zurückgegriffen werden.

\subsubsection*{13.92}
\paragraph{(a)} Wir möchten in dieser Aufgabe ein Signal von \SI{0.25}{\volt} um $\SI{26}{\decibel} = 19.95$ verstärken. Entscheidend ist hierbei die Kleinsignalbedingung. Diese gilt bei einem BJT nur bis \SI{0.005}{\volt}. Da hier allerdings ein deutlich stärkeres Signal verstärkt werden soll, bietet sich für diese Aufgabe ein MOSFET besser an.
\paragraph{(b)} Auch hier aus gleichem Grund bietet sich ein MOSFET besser an, da wir Eingangssignale größer \SI{0.005}{\volt} sehen.

\subsection{The Common-Source Amplifier}
\subsubsection*{13.93}
Ein CS-Verstärker mit $(V_{GS}-V_{TN} = \SI{1}{\volt})$ arbeitet mit einer Versorgungsspannung von \SI{15}{\volt}. Wie groß ist die geschätzte Verstärkung?\\
Mithilfe von Gleichung (13.91), welche aus einigen Vereinfachungen hervorgeht, können wir die Verstärkung sehr leicht bestimmen:
\[ A_v^{CS} = -\frac{V_{DD}}{V_{GS}-V_{TN}} = -V_{DD} = -\SI{15}{\volt}\]

\subsubsection*{13.94}
Ein CS-Verstärker hat eine Verstärkung von $\SI{16}{\decibel} = 6.3$ und entwickelt an seinem Ausgang ein Signal von $V_{PP} = \SI{15}{\volt}$. Arbeitet dieser Verstärker in seiner Kleinsignal-Region?\\
Wie groß das ursprüngliche Signal war, können wir durch Rückrechnung bestimmen:
\[ v_{in} = \frac{\SI{7.5}{\volt}}{6.3} = \SI{1.19}{\volt} \]
Mit der Kleinsignalbedingung $v_i \leq 0.2(V_{GS}-V_{TN})$ muss gelten $V_{GS}-V_{TN} \geq \SI{5.95}{\volt}$.

\subsubsection*{13.95}
Ein CS-Verstärker ($K_n = \SI{1}{\milli\ampere\per\square\volt}$) wird mit einer Versorgungsspannung von \SI{18}{\volt} betrieben. Welcher Arbeitsstrom $I_D$ wird für eine Verstärkung von 30 benötigt?
\[ A_v^{CS} = 30 = - \frac{V_{DD}}{V_{GS}-V_{TN}} \Leftrightarrow V_{GS}-V_{TN} = - \frac{\SI{18}{\volt}}{30} = - \SI{0.6}{\volt} \]
Aus der Beziehung (siehe Tabelle 13.3)
\[ \frac{2I_D}{V_{GS}-V_{TN}} = \sqrt{2K_nI_D} \]
kann schnell folgender Zusammenhang gebildet werden:
\[ V_{GS}-V_{TN} = \frac{2I_D}{\sqrt{2K_nI_D}} = \sqrt{\frac{2I_D}{K_n}} = -\SI{0.6}{\volt} \]
Löst man letzte Gleichung nach $I_D$ auf, erhalten wir:
\[ I_D = \frac{(\SI{-0.6}{\volt})^2K_n}{2} = \SI{180}{\micro\ampere} \]

\subsubsection*{13.96}
Ein CS-Verstärker arbeitet mit einer Versorgungsspannung von \SI{9}{\volt}. Was ist der maximale Wert für $V_{GS}-V_{TN}$, wenn eine Verstärkung von 25 erreicht werden soll?\\
Zunächst stellen wir die allgemeine Gleichung für die Verstärkung auf:
\[  A_v^{CS} = 25 = - \frac{V_{DD}}{V_{GS}-V_{TN}} \Leftrightarrow V_{GS}-V_{TN} = - \frac{V_{DD}}{25} = - \frac{\SI{9}{\volt}}{25} = - \SI{0.36}{\volt}\]
Damit wurde gezeigt, dass $V_{GS}-V_{TN}$ maximal \SI{0.36}{\volt} betragen werden darf.

\subsubsection*{13.97}
Ein CS-Verstärker soll ein Wechselsignal $v_{gs}$ der Höhe \SI{0.2}{\volt} verstärken. Welchen Wert soll $V_{GS}-V_{TN}$ minimal haben, um die Kleinsignalbedingung nicht zu verletzen?\\
Für MOSFETs gilt folgende Kleinsignalbedingung:
\[ v_{gs} \leq 0.2(V_{GS}-V_{TN}) \Leftrightarrow V_{GS}-V_{TN} \geq \SI{1}{\volt} \]
Wenn eine Verstärkung von \SI{33}{\decibel} erreicht werden soll, muss folgende Gleichung gelten:
\[ A_{v}^{CS} = 10^{\frac{33}{20}} = 44.67 \approx \frac{V_{DD}}{V_{GS}-V_{TN}} \Leftrightarrow V_{DD} = 44.67(V_{GS}-V_{TN}) = \SI{44.67}{\volt} \]

\subsubsection*{13.98}
Ein CS-Verstärker soll ein Wechselsignal $v_{gs}$ der Höhe \SI{0.4}{\volt} verstärken. Welchen Wert soll $V_{GS}-V_{TN}$ minimal haben, um die Kleinsignalbedingung nicht zu verletzen?\\
Für MOSFETs gilt folgende Kleinsignalbedingung:
\[ v_{gs} \leq 0.2(V_{GS}-V_{TN}) \Leftrightarrow V_{GS}-V_{TN} \geq \SI{2}{\volt} \]
Wenn eine Verstärkung von \SI{26}{\decibel} erreicht werden soll, muss folgende Gleichung gelten:
\[ A_{v}^{CS} = 10^{\frac{26}{20}} = 19.95 \approx \frac{V_{DD}}{V_{GS}-V_{TN}} \Leftrightarrow V_{DD} = 19.95(V_{GS}-V_{TN}) = \SI{39.9}{\volt} \]

\subsubsection*{13.99}
Es ist ein Verstärker mit der Verstärkung von 1000 gesucht. Er soll mit einer Kaskade von CS-Verstärkern gestaltet werden, welche mit einer Versorgungsspannung von \SI{12}{\volt} betrieben werden. Wie viele CS-Bausteine benötigt man?
Schätzt man die Verstärkung eines einzelnen Bausteins mit 
\[ A_v^{CS} = -V_{DD} \]
ab, kann die Anzahl $x$ folgendermaßen bestimmt werden:
\[ V_{DD}^x \geq 1000 \Leftrightarrow x \geq \log_{V_{DD}}(1000) = 2.78 \]
Es werden also minimal drei CS-Verstärker benötigt, um eine Verstärkung von 1000 zu erreichen.

\subsubsection*{13.100}
Um den Text hier klein zu halten, führe ich die vorgegebenen Werte hier nicht explizit auf, sondern verarbeite sie direkt in den Rechnungen.\\
\begin{enumerate}
	\item \textbf{DC Arbeitspunkt}: Um die Transkonduktanz für die Verstärkung bestimmen zu können, benötigt man den Arbeitspunkt im dc-Betrieb. Diesen erhält man, in dem alle Kondensatoren aufgeschnitten  und ac-Quellen kurzgeschaltet werden. Anschließend kann die Ersatzquelle und der Ersatzwiderstand bestimmt werden:
	\begin{equation}
	\begin{aligned}
	  V_{th} &= V_{DD}\frac{R_1}{R_1+R_2} = \SI{10}{\volt} \frac{\SI{430}{\kilo\ohm}}{\SI{430}{\kilo\ohm} + \SI{560}{\kilo\ohm}} = \SI{4.34}{\volt}\\
	  R_{th} &= R_1 || R_2 = \frac{R_1R_2}{R_1+R_2} = \frac{(\SI{430}{\kilo\ohm})(\SI{560}{\kilo\ohm})}{\SI{430}{\kilo\ohm}+\SI{560}{\kilo\ohm}} = \SI{243.32}{\kilo\ohm}
	\end{aligned}
	\end{equation}
	Nun können wir ein Gleichungssystem aufstellen, um $I_D$ zu bestimmen. Dabei wir nehmen wir an, dass der Transistor in der aktiven Region arbeitet:
	\begin{equation}
	\begin{aligned}
	V_{GS} &= V_{th} - I_DR_4 \\
	I_D &= \frac{K_n}{2}(V_{GS}-V_{TN})^2(1+\lambda V_{DS}) \\
	    &= \frac{K_n}{2}(V_{GS}-V_{TN})^2(1+\lambda (V_{DD}-R_DI_D-R_4I_D))
	\end{aligned}
	\end{equation}
	Setzt man nun die erste Gleichung in die zweite ein, erhält man:
	\begin{equation}
	\begin{aligned}
	I_D &= \frac{K_n}{2}(V_{th} - I_DR_4-V_{TN})^2(1+\lambda (V_{DD}-R_DI_D-R_4I_D))\\
	I_{D1} &= \SI{1.34}{\milli\ampere}\\
	I_{D2} &= \SI{129.5}{\micro\ampere}
	\end{aligned}
	\end{equation}
	Schaut man sich beide Lösungen an, kommt man schnell auf den Schluss, dass lediglich $I_{D2}$ die korrekte Lösung sein kann.\\
	Da zu einem späteren Zeitpunkt in diesem Problemset die Spannung $V_{DS}$ benötigt wird, können wir diese noch bestimmen:
	\[ V_{DS} = V_{DD} - R_DI_D - R_4I_D = \SI{10}{\volt} - (\SI{43}{\kilo\ohm})(\SI{129.5}{\micro\ampere}) - (\SI{20}{\kilo\ohm})(\SI{129.5}{\micro\ampere}) = \SI{1.84}{\volt} \]
	
	\item \textbf{AC-Analyse}: Mit dem Arbeitsstrom $I_D$ kann nun die Transkonduktanz bestimmt werden.
	\[ g_m=\sqrt{2K_nI_D} = \sqrt{2(\SI{0.45}{\milli\ampere\per\square\volt})(\SI{129.5}{\micro\ampere})} = \SI{0.34}{\milli\siemens} \]
	Nun kann der Last- und Eingangswiderstand berechnet werden:
	\begin{equation}
	\begin{aligned}
	R_G &= R_1||R_2 = \SI{243.32}{\kilo\ohm}\\
	R_L &= r_0||R_D||R_3 = \frac{1}{\lambda I_D} || \SI{43}{\kilo\ohm} || \SI{100}{\kilo\ohm} \approx \SI{28.6}{\kilo\ohm}
	\end{aligned}
	\end{equation}
	Nun kann zu guter Letzt die Verstärkergleichung benutzt werden:
	\[ A_v^{CS} = -g_mR_L\frac{R_G}{R_G+R_I} = -(\SI{0.34}{\milli\siemens})(\SI{28.6}{\kilo\ohm}) \frac{\SI{243.32}{\kilo\ohm}}{\SI{243.32}{\kilo\ohm} + \SI{1}{\kilo\ohm}} = -9.68 \]
	
\end{enumerate}

\subsubsection*{13.101}
Das Gleiche wie in der vorherigen Aufgabe, nur dass uns dieses Mal die DC-Analyse erspart bleibt! Wir können also die Transkonduktanz direkt mit
\[ g_m = \sqrt{2K_nI_D} = \sqrt{(2)(\SI{450}{\micro\ampere\per\square\volt})(\SI{100}{\micro\ampere})} = \SI{0.3}{\milli\siemens} \] 
angegeben werden.\\
Da es sich hier prinzipiell um die gleiche Schaltung wie in (13.100) handelt, können wir auch exakt den gleichen Rechenweg anwenden. Deshalb bestimmen wir zunächst den Lastwiderstand $R_L$:
\[ R_L = r_0||R_D||R_3 = \frac{1}{\lambda I_D} || \SI{50}{\kilo\ohm} || \SI{120}{\kilo\ohm} \approx \SI{33}{\kilo\ohm} \]
Jetzt kann die CS-Verstärkung folgendermaßen berechnet werden:
\[ A_v^{CS} = -g_mR_L\frac{R_G}{R_G+R_I} = -(\SI{0.3}{\milli\siemens})(\SI{33}{\kilo\ohm}) \frac{\SI{6.8}{\mega\ohm}}{\SI{6.8}{\mega\ohm} + \SI{100}{\kilo\ohm}} \approx - 9.76\]
Hinweis: Diese Aufgabe kann auch genauer berechnet werden, wenn noch die Info zu $V_{DS}$ genutzt wird. Dann kann auch beispielsweise die genauere Formel für den Ausgangswiderstand $r_0$ und die Transkonduktanz $g_m = \sqrt{2K_nI_D(1+\lambda V_{DS})}$ verwendet werden.

\subsubsection*{13.102}
Diese Aufgabe lässt sich leicht lösen, da die einzige Abhängigkeit von $K_n$ herrscht in der Transkonduktanz $g_m$. So können wir einfach den Minimal- und Maximalwert berechnen und anschließend in die Verstärkung-Gleichung einsetzen:
\begin{equation}
\begin{aligned}
g_m^{\text{max}} &= \sqrt{2K_n^{\text{max}}I_D} = \sqrt{2(\SI{700}{\micro\ampere\per\square\volt})(\SI{100}{\micro\ampere})} = \SI{374}{\micro\siemens} \\
g_m^{\text{min}} &= \sqrt{2K_n^{\text{min}}I_D} = \sqrt{2(\SI{300}{\micro\ampere\per\square\volt})(\SI{100}{\micro\ampere})} = \SI{245}{\micro\siemens} \\
A_v^{\text{CS, max}} &= -g_m^{\text{max}}R_L\frac{R_G}{R_G+R_I} = -(\SI{374}{\micro\siemens})(\SI{33}{\kilo\ohm})\frac{\SI{6.8}{\mega\ohm}}{\SI{6.8}{\mega\ohm}+\SI{100}{\kilo\ohm}} = -12.16\\
A_v^{\text{CS, min}} &= -g_m^{\text{min}}R_L\frac{R_G}{R_G+R_I} = -(\SI{245}{\micro\siemens})(\SI{33}{\kilo\ohm})\frac{\SI{6.8}{\mega\ohm}}{\SI{6.8}{\mega\ohm}+\SI{100}{\kilo\ohm}} = -7.97
\end{aligned}
\end{equation}

\subsubsection*{13.103}
Die Aufgabe ist äquivalent zu (13.101) – mit gleichem Arbeitspunkt aber mit wesentlich höherer Last. Dieses Mal werden wir allerdings eine genauere Berechnung durchführen Zuerst die Transkonduktanz:
\[ g_m = \sqrt{2K_nI_D(1+\lambda V_{DS})} = \sqrt{2(\SI{100}{\micro\ampere\per\square\volt})(\SI{10}{\micro\ampere})(1+\SI{0.02}{\per\volt}\SI{5}{\volt})} = \SI{46.9}{\micro\siemens} \] 
Als zweites den Ausgangswiderstand und der Lastwiderstand:
\begin{equation} 
\begin{aligned}
r_0 &= \frac{\frac{1}{\lambda}+V_{DS}}{I_D} = \frac{\frac{1}{\SI{0.02}{\per\volt}}+\SI{5}{\volt}}{\SI{10}{\micro\ampere}} = \SI{5.5}{\mega\ohm}\\
R_L &= r_0||R_D||R_3 = \SI{5.5}{\mega\ohm} || \SI{560}{\kilo\ohm} || \SI{1.5}{\mega\ohm} \approx \SI{379.6}{\kilo\ohm}
\end{aligned}
\end{equation}
Und zuletzt die Verstärkung:
\[ A_v^{CS} = -g_mR_L\frac{R_G}{R_G+R_I} = -(\SI{46.9}{\micro\siemens})(\SI{379.6}{\kilo\ohm}) \frac{\SI{10}{\mega\ohm}}{\SI{10}{\mega\ohm} + \SI{100}{\kilo\ohm}} \approx - 17.63 \]

\subsubsection*{13.104}
Gleicher Aufgabentyp zu (13.101) und (13.103). Keine weiteren Bemerkungen.
\begin{equation} 
\begin{aligned}
g_m &= \sqrt{2K_nI_D(1+\lambda V_{DS})} = \sqrt{2(\SI{1}{\milli\ampere\per\square\volt})(\SI{2}{\milli\ampere})(1+\SI{0.015}{\per\volt}\SI{7.5}{\volt})} = \SI{2.11}{\milli\siemens} \\
r_0 &= \frac{\frac{1}{\lambda}+V_{DS}}{I_D} = \frac{\frac{1}{\SI{0.015}{\per\volt}}+\SI{7.5}{\volt}}{\SI{2}{\milli\ampere}} \approx \SI{37}{\kilo\ohm}\\
R_L &= r_0||R_D||R_3 = \SI{37}{\kilo\ohm} || \SI{3.9}{\kilo\ohm} || \SI{33}{\kilo\ohm} \approx \SI{3.2}{\kilo\ohm} \\
A_v^{CS} &= -g_mR_L\frac{R_G}{R_G+R_I} = -(\SI{2.11}{\milli\siemens})(\SI{3.2}{\kilo\ohm}) \frac{\SI{1}{\mega\ohm}}{\SI{1}{\mega\ohm} + \SI{10}{\kilo\ohm}} \approx - 6.34
\end{aligned}
\end{equation}

\subsection*{Input and Output Resistances of the Common-Emitter and Common-Source Amplifiers}
\subsubsection*{13.109}
Berechne für den abgebildeten Schaltplan den Eingangs- und Ausgangswiderstand, wenn der Q-Punkt bei $(\SI{75}{\micro\ampere}, \SI{10}{\volt})$ liegt. Nimm $\beta_0=100$ und $V_A=\SI{75}{\volt}$ an.\\
Baut man das Kleinsignalmodell des BJT in die Schaltung, kann der Eingangswiderstand schnell gefunden werden. Zuerst die Berechnung des Transistor-Parameters:
\[ r_{\pi} = \frac{\beta_0V_T}{I_C} = \frac{(100)(\SI{0.025}{\volt})}{\SI{75}{\micro\ampere}} = \SI{33.3}{\kilo\ohm}\]
Um letztlich die Parallelschaltung lösen zu können:
\[ R_{\text{in}}^{CE} = R_B || r_{\pi} =  \frac{(\SI{100}{\kilo\ohm})(\SI{33.3}{\kilo\ohm})}{\SI{100}{\kilo\ohm}+\SI{33.3}{\kilo\ohm}} \approx \SI{25}{\kilo\ohm}\]
Als nächstes folgt die Berechnung des Ausgangswiderstandes. Diese ist ganz einfach, da lediglich in den Ausgang geschaut werden muss. Doch vorher muss noch der Transistor-eigene Ausgangswiderstand $r_{\text{o}}$ berechnet werden:
\[ r_{\text{o}} = \frac{V_A+V_{CE}}{I_C} = \frac{\SI{75}{\volt} + \SI{10}{\volt}}{\SI{75}{\micro\ampere}} = \SI{1.13}{\mega\ohm} \]
Nun kann der eigentliche Ausgangswiderstand bestimmt werden:
\[ R_{\text{out}}^{CE} = R_C || r_{\text{o}} = \frac{(\SI{100}{\kilo\ohm})(\SI{1.13}{\mega\ohm})}{\SI{100}{\kilo\ohm} + \SI{1.13}{\mega\ohm}} \approx \SI{92}{\kilo\ohm}  \]

\subsubsection*{13.110}
Exakt gleicher Aufgabentypus wie in (13.109). Wir müssen lediglich andere Zahlenwerte einsetzen:
\begin{equation}
\begin{aligned}
r_{\pi} &= \frac{\beta_0V_T}{I_C} = \frac{(75)(\SI{0.025}{\volt})}{\SI{2}{\milli\ampere}} = \SI{937.5}{\ohm} \\
R_{\text{in}}^{CE} &= R_B || r_{\pi} =  \frac{(\SI{4.7}{\kilo\ohm})(\SI{937.5}{\ohm})}{\SI{4.7}{\kilo\ohm}+\SI{937.5}{\ohm}} \approx \SI{781.6}{\ohm} \\
r_{\text{o}} &= \frac{V_A+V_{CE}}{I_C} = \frac{\SI{50}{\volt} + \SI{7.5}{\volt}}{\SI{2}{\milli\ampere}} = \SI{28.75}{\kilo\ohm} \\
R_{\text{out}}^{CE} &= R_C || r_{\text{o}} = \frac{(\SI{4.3}{\kilo\ohm})(\SI{28.75}{\kilo\ohm})}{\SI{4.3}{\kilo\ohm} + \SI{28.75}{\kilo\ohm}} \approx \SI{3.74}{\kilo\ohm}
\end{aligned}
\end{equation}

\subsubsection*{13.111}
In Problem (13.59) beläuft sich die Vorwärts-Verstärkung $\beta_0$ in einem Bereich von 60 bis 100. Welche Minimal- und Maximalwerte kann der Eingangs- und Ausgangswiderstand erreichen?\\
Zuerst muss die Abhängigkeit der Widerstände festgestellt werden. Hierbei stellt sich heraus, dass der Ausgangswiderstand gar nicht von $\beta_0$ abhängig ist und damit einen festen Wert aufweist. Der Transistor-Eingangswiderstand $r_{\pi}$ dagegen ist proportional zu $\beta_0$: $r_{\pi} = \frac{\beta_0V_T}{I_C}$.
\begin{equation}
\begin{aligned}
r_{\pi}^{\text{min}} &= \frac{\beta_0^{\text{min}}V_T}{I_C} = \frac{(60)(\SI{0.025}{\volt})}{\SI{40}{\micro\ampere}} = \SI{37.5}{\kilo\ohm} \\
r_{\pi}^{\text{max}} &= \frac{\beta_0^{\text{max}}V_T}{I_C} = \frac{(100)(\SI{0.025}{\volt})}{\SI{40}{\micro\ampere}} = \SI{62.5}{\kilo\ohm}
\end{aligned}
\end{equation} 
Nun können die Ausreißer-Werte für den Gesamteingangswiderstand berechnet werden:
\begin{eqnarray}
\begin{aligned}
R_{\text{in}}^{CE\text{, min}} &= R_B || r_{\pi}^{\text{min}} = \frac{(\SI{100}{\kilo\ohm})(\SI{37.5}{\kilo\ohm})}{\SI{100}{\kilo\ohm}+\SI{37.5}{\kilo\ohm}} \approx \SI{27.3}{\kilo\ohm} \\
R_{\text{in}}^{CE\text{, max}} &= R_B || r_{\pi}^{\text{max}} = \frac{(\SI{100}{\kilo\ohm})(\SI{62.5}{\kilo\ohm})}{\SI{100}{\kilo\ohm}+\SI{62.5}{\kilo\ohm}} \approx \SI{38.5}{\kilo\ohm}
\end{aligned}
\end{eqnarray}
Zu guter Letzt berechnen wir den von $\beta_0$ unabhängigen Ausgangswiderstand:
\begin{equation}
\begin{aligned}
r_{\text{o}} &= \frac{V_A+V_{CE}}{I_C} = \frac{\SI{75}{\volt} + \SI{10}{\volt}}{\SI{40}{\micro\ampere}} = \SI{2.125}{\mega\ohm} \\
R_{\text{out}}^{CE} &= R_C || r_{\text{o}} = \frac{(\SI{62}{\kilo\ohm})(\SI{2.125}{\mega\ohm})}{\SI{62}{\kilo\ohm}+\SI{2.125}{\mega\ohm}} = \SI{60.24}{\kilo\ohm}
\end{aligned}
\end{equation}

\subsubsection*{13.112}
Exakt gleicher Aufgabentypus wie in (13.109). Wir müssen lediglich andere Zahlenwerte einsetzen:
\begin{equation}
\begin{aligned}
r_{\pi} &= \frac{\beta_0V_T}{I_C} = \frac{(40)(\SI{0.025}{\volt})}{\SI{1}{\micro\ampere}} = \SI{1}{\mega\ohm} \\
R_{\text{in}}^{CE} &= R_B || r_{\pi} =  \frac{(\SI{5}{\mega\ohm})(\SI{1}{\mega\ohm})}{\SI{5}{\mega\ohm}+\SI{1}{\mega\ohm}} \approx \SI{833.3}{\kilo\ohm} \\
r_{\text{o}} &= \frac{V_A+V_{CE}}{I_C} = \frac{\SI{50}{\volt} + \SI{1.5}{\volt}}{\SI{1}{\micro\ampere}} = \SI{51.5}{\mega\ohm} \\
R_{\text{out}}^{CE} &= R_C || r_{\text{o}} = \frac{(\SI{2}{\mega\ohm})(\SI{51.5}{\mega\ohm})}{\SI{2}{\mega\ohm} + \SI{51.5}{\mega\ohm}} \approx \SI{1.9}{\mega\ohm}
\end{aligned}
\end{equation}

\subsubsection*{13.113}
\textbf{(a)} $5 \times I_C$, $\frac{1}{5} \times R_C,R_B,R_3$ 
\begin{equation}
\begin{aligned}
r_{\pi} &= \frac{\beta_0V_T}{I_C} = \frac{(40)(\SI{0.025}{\volt})}{\SI{5}{\micro\ampere}} = \SI{200}{\kilo\ohm} \\
R_{\text{in}}^{CE} &= R_B || r_{\pi} =  \frac{(\SI{1}{\mega\ohm})(\SI{200}{\kilo\ohm})}{\SI{1}{\mega\ohm}+\SI{200}{\kilo\ohm}} \approx \SI{166.7}{\kilo\ohm} \\
r_{\text{o}} &= \frac{V_A+V_{CE}}{I_C} = \frac{\SI{50}{\volt} + \SI{1.5}{\volt}}{\SI{5}{\micro\ampere}} = \SI{10.3}{\mega\ohm} \\
R_{\text{out}}^{CE} &= R_C || r_{\text{o}} = \frac{(\SI{400}{\kilo\ohm})(\SI{10.3}{\mega\ohm})}{\SI{400}{\kilo\ohm} + \SI{10.3}{\mega\ohm}} \approx \SI{385}{\kilo\ohm}
\end{aligned}
\end{equation}
\textbf{(b)} $50 \times I_C$, $\frac{1}{50} \times R_C,R_B,R_3$
\begin{equation}
\begin{aligned}
r_{\pi} &= \frac{\beta_0V_T}{I_C} = \frac{(40)(\SI{0.025}{\volt})}{\SI{50}{\micro\ampere}} = \SI{20}{\kilo\ohm} \\
R_{\text{in}}^{CE} &= R_B || r_{\pi} =  \frac{(\SI{100}{\kilo\ohm})(\SI{20}{\kilo\ohm})}{\SI{100}{\kilo\ohm}+\SI{20}{\kilo\ohm}} \approx \SI{16.7}{\kilo\ohm} \\
r_{\text{o}} &= \frac{V_A+V_{CE}}{I_C} = \frac{\SI{50}{\volt} + \SI{1.5}{\volt}}{\SI{50}{\micro\ampere}} = \SI{1.03}{\mega\ohm} \\
R_{\text{out}}^{CE} &= R_C || r_{\text{o}} = \frac{(\SI{40}{\kilo\ohm})(\SI{1.03}{\mega\ohm})}{\SI{40}{\kilo\ohm} + \SI{1.03}{\mega\ohm}} \approx \SI{38.5}{\kilo\ohm}
\end{aligned}
\end{equation}

\subsubsection*{13.114}
Um Eingangs- und Ausgangswiderstand berechnen zu können, benötigen wir den in Aufgabe (13.100) berechneten Arbeitspunkt: $(\SI{129.5}{\micro\ampere},\SI{1.84}{\volt})$. Nun arbeiten wir mit den gewohnten Gleichungen aus vorherigen Aufgaben:
\begin{equation}
\begin{aligned}
R_{\text{in}}^{CS} &= R_1 || R_2 = \frac{(\SI{430}{\kilo\ohm})(\SI{560}{\kilo\ohm})}{\SI{430}{\kilo\ohm} + \SI{560}{\kilo\ohm}} \approx \SI{243}{\kilo\ohm} \\
r_{\text{o}} &= \frac{\frac{1}{\lambda}+V_{DS}}{I_D} = \frac{\SI{77}{\volt}}{\SI{129.5}{\micro\ampere}} = \SI{594.8}{\kilo\ohm} \\
R_{\text{out}}^{CE} &= R_D || r_{\text{o}} = \frac{(\SI{43}{\kilo\ohm})(\SI{594.8}{\kilo\ohm})}{\SI{43}{\kilo\ohm} + \SI{594.8}{\kilo\ohm}} = \SI{40.1}{\kilo\ohm}
\end{aligned}
\end{equation}

\subsubsection*{13.115}
Um Eingangs- und Ausgangswiderstand berechnen zu können, benötigen wir den in Aufgabe (13.101) angegebenen Arbeitspunkt: $(\SI{100}{\micro\ampere},\SI{5}{\volt})$. Nun arbeiten wir mit den gewohnten Gleichungen aus vorherigen Aufgaben:
\begin{equation}
\begin{aligned}
R_{\text{in}}^{CS} &= R_G = \SI{6.8}{\mega\ohm}\\
r_{\text{o}} &= \frac{\frac{1}{\lambda}+V_{DS}}{I_D} = \frac{\SI{55}{\volt}}{\SI{100}{\micro\ampere}} = \SI{550}{\kilo\ohm} \\
R_{\text{out}}^{CE} &= R_D || r_{\text{o}} = \frac{(\SI{50}{\kilo\ohm})(\SI{550}{\kilo\ohm})}{\SI{50}{\kilo\ohm} + \SI{550}{\kilo\ohm}} = \SI{45.8}{\kilo\ohm}
\end{aligned}
\end{equation}

\subsubsection*{13.116}
Wie bereits in der vorherigen Aufgabe (13.115) ersichtlich, sind weder Eingangs- noch Ausgangswiderstand von $K_n$ abhängig. Dementsprechend gelten die gleichen Ergebnisse wie in (13.115).

\subsubsection*{13.117}
Der Eingangswiderstand kann für einen CS-Verstärker sehr schnell berechnet werden:
\[ R_{\text{in}}^{CS} = R_G = \SI{10}{\mega\ohm} \]
Der Ausgangswiderstand kann wie in den vorherigen Aufgaben folgendermaßen berechnet werden:
\begin{equation}
\begin{aligned}
r_{\text{o}} &= \frac{\frac{1}{\lambda}+V_{DS}}{I_D} = \frac{\SI{55}{\volt}}{\SI{10}{\micro\ampere}} = \SI{5.5}{\mega\ohm} \\
R_{\text{out}}^{CE} &= R_D || r_{\text{o}} = \frac{(\SI{560}{\kilo\ohm})(\SI{5.5}{\mega\ohm})}{\SI{560}{\kilo\ohm} + \SI{5.5}{\mega\ohm}} = \SI{508.25}{\kilo\ohm}
\end{aligned}
\end{equation}

\subsubsection*{13.118}
\begin{equation}
\begin{aligned}
R_{\text{in}}^{CS} &= R_G = \SI{1}{\mega\ohm}\\
r_{\text{o}} &= \frac{\frac{1}{\lambda}+V_{DS}}{I_D} = \frac{\SI{74.2}{\volt}}{\SI{2}{\milli\ampere}} = \SI{37}{\kilo\ohm} \\
R_{\text{out}}^{CE} &= R_D || r_{\text{o}} = \frac{(\SI{3.9}{\kilo\ohm})(\SI{37}{\kilo\ohm})}{\SI{3.9}{\kilo\ohm} + \SI{37}{\kilo\ohm}} = \SI{3.5}{\kilo\ohm}
\end{aligned}
\end{equation}

\subsubsection*{13.119}
Neuer Aufgabentyp: Berechne das Thévenin-Äquivalent der Schaltung aus Prob. 13.59.\\
Dazu ersetzen wir den Transistor durch sein Kleinsignalmodell und errechnen, wie viel der Eingangsspannung an den Widerständen am Ausgang ankommen. 
\begin{equation}
	\begin{aligned}
	g_m &= \frac{I_C}{V_T} = \frac{\SI{40}{\micro\ampere}}{\SI{0.025}{\volt}} = \SI{1.6}{\milli\siemens} \\
	r_{\pi} &= \frac{\beta_0}{g_m} = \frac{100}{\SI{1.6}{\milli\siemens}} = \SI{62.5}{\kilo\ohm} \\
	r_o &= \frac{V_A+V_{CE}}{I_C} = \frac{\SI{75}{\volt} + \SI{10}{\volt}}{\SI{40}{\micro\ampere}} = \SI{2.125}{\mega\ohm} \\
	v_{\text{th}} &= -g_mv_{be}r_o||R_C = -g_mv_i\frac{R_B||R_{\pi}}{R_I + R_B||R_{\pi}}r_o||R_C = -94.4v_i
	\end{aligned}
\end{equation}
Zum Schluss schaut man in die Ausgangsports und berechnet den Ersatzwiderstand:
\[ R_{th} = R_C||r_o = \SI{60.2}{\kilo\ohm} \]

\subsubsection*{13.120}
Nochmal eine Thévenin-Darstellung für Aufgabe 13.61.
\begin{equation}
\begin{aligned}
g_m &= \frac{I_C}{V_T} = \frac{\SI{2}{\milli\ampere}}{\SI{0.025}{\volt}} = \SI{80}{\milli\siemens} \\
r_{\pi} &= \frac{\beta_0}{g_m} = \frac{75}{\SI{80}{\milli\siemens}} = \SI{937.5}{\ohm} \\
r_o &= \frac{V_A+V_{CE}}{I_C} = \frac{\SI{50}{\volt} + \SI{7.5}{\volt}}{\SI{2}{\milli\ampere}} = \SI{28.75}{\kilo\ohm} \\
v_{\text{th}} &= -g_mv_{be}r_o||R_C = -g_mv_i\frac{R_B||R_{\pi}}{R_I + R_B||R_{\pi}}r_o||R_C = -303.8v_i \\
R_{th} &= R_C||r_o = \SI{4039.6}{\ohm}
\end{aligned}
\end{equation}

\subsubsection*{13.121}
Nochmal eine Thévenin-Darstellung für Aufgabe 13.103.
\begin{equation}
\begin{aligned}
g_m &= \sqrt{2K_nI_D(1+\lambda V_{DS})} = \sqrt{2(\SI{100}{\micro\ampere\per\square\volt})(\SI{10}{\micro\ampere})(1+(\SI{0.02}{\per\volt})(\SI{5}{\volt})} = \SI{46.9}{\micro\siemens} \\
r_o &= \frac{\frac{1}{\lambda}+V_{DS}}{I_D} = \frac{\frac{1}{\SI{0.02}{\per\volt}}+\SI{5}{\volt}}{\SI{10}{\micro\ampere}} = \SI{5.5}{\mega\ohm} \\
v_{\text{th}} &= -g_mv_{gs}r_o||R_D = -g_mv_i\frac{R_G}{R_I + R_G}r_o||R_D = -23.6v_i \\
R_{th} &= R_D||r_o = \SI{508.25}{\kilo\ohm}
\end{aligned}
\end{equation}

\subsubsection*{13.122}
Nochmal eine Thévenin-Darstellung für Aufgabe 13.101.
\begin{equation}
\begin{aligned}
g_m &= \sqrt{2K_nI_D(1+\lambda V_{DS})} = \sqrt{2(\SI{450}{\micro\ampere\per\square\volt})(\SI{100}{\micro\ampere})(1+(\SI{0.02}{\per\volt})(\SI{5}{\volt})} = \SI{314.6}{\micro\siemens} \\
r_o &= \frac{\frac{1}{\lambda}+V_{DS}}{I_D} = \frac{\frac{1}{\SI{0.02}{\per\volt}}+\SI{5}{\volt}}{\SI{100}{\micro\ampere}} = \SI{550}{\kilo\ohm} \\
v_{\text{th}} &= -g_mv_{gs}r_o||R_D = -g_mv_i\frac{R_G}{R_I + R_G}r_o||R_D = -14.2v_i \\
R_{th} &= R_D||r_o = \SI{45.8}{\kilo\ohm}
\end{aligned}
\end{equation}

\subsection{Common-Emitter and Common-Source Amplifier Summary}
\subsubsection*{13.123}
\paragraph{(a)} Dieser Aufgabentyp kombiniert alle vorherigen Aufgaben in einer: Es sollen Verstärkung, Eingangs- und Ausgangswiderstand berechnet werden. Um nochmal alles frisch im Kopf aufzubauen und sich von alten Routinen zu lösen, wird für die Lösung dieser Aufgabe ein kompletter Hergang aufgezeigt.\\
Im ersten Schritt berechnen wir den DC-Arbeitspunkt. Mit diesem können alle Kleinsignalparameter de Transistors berechnet werden. Mit dem Kleinsignalmodell lassen sich Verstärkung, Eingangs- und Ausgangswiderstand berechnen.\\
\begin{enumerate}
	\item \textbf{Bestimmung des DC-Arbeitspunktes}\\
	Wir beginnen mit einer einfachen Maschengleichung am Eingang:
	\begin{equation}
	\begin{aligned}
	&&0 &= I_B(\SI{100}{\kilo\ohm}) + V_{BE} + I_E(\SI{16}{\kilo\ohm}) - \SI{5}{\volt} \\
	&\Leftrightarrow &I_B  &= -\frac{\SI{0.7}{\volt} - \SI{5}{\volt}}{\SI{100}{\kilo\ohm} + (1+\beta_f)(\SI{16}{\kilo\ohm})} \\
	&&&= \SI{3.72}{\micro\ampere}
	\end{aligned}
	\end{equation}
	Mit dem berechneten Basisstrom können wir den vollständigen Arbeitspunkt bestimmen:
	\begin{equation}
	\begin{aligned}
	I_C &= \beta_fI_B = (65)(\SI{3.72}{\micro\ampere}) = \SI{241.78}{\micro\ampere}\\
	V_{CE} &= \SI{10}{\volt} - I_C(\SI{10}{\kilo\ohm}) - I_E(\SI{16}{\kilo\ohm}) \\
	       &= \SI{3.65}{\volt}
	\end{aligned}
	\end{equation}
	Damit lautet der Arbeitspunkt (\SI{241.78}{\micro\ampere}, \SI{3.65}{\volt}).
	
	\item \textbf{Bestimmung der Kleinsignalparameter}\\
	Um das Kleinsignalmodell benutzen zu können, müssen noch Größen der Bauteile und die Transkonduktanz berechnet werden.
	\begin{equation}
	\begin{aligned}
	   g_m &= \frac{I_C}{V_T} = \frac{\SI{241.78}{\micro\ampere}}{\SI{0.025}{\volt}} = \SI{9.67}{\milli\siemens} \\
	   r_{\pi} &= \frac{\beta_0}{g_m} = \frac{65}{\SI{9.67}{\milli\siemens}} = \SI{6721}{\ohm} \\
	   r_o &= \frac{V_A + V_{CE}}{I_C} = \frac{\SI{53.65}{\volt}}{\SI{241.78}{\micro\ampere}} = \SI{221.9}{\kilo\ohm}
	\end{aligned}
	\end{equation}
	\item \textbf{Bestimmung der Verstärkung} \\
	Die Spannungsverstärkung bestimmen wir über mehrere Spannungsteiler und das Ohm'sche Gesetz.
	\begin{equation}
	\begin{aligned}
	  v_{be} &= v_i\frac{\SI{10}{\kilo\ohm}||r_{\pi}}{\SI{330}{\ohm} + \SI{10}{\kilo\ohm}||r_{\pi}} = 0.924v_i \\
	  A_v^{CS} &= - \frac{v_{be}}{v_i}g_mr_o||\SI{10}{\kilo\ohm}||\SI{220}{\kilo\ohm} = -(0.924)(\SI{9.67}{\milli\siemens})(\SI{9169.9}{\ohm}) = -81.9
	\end{aligned}
	\end{equation}
	
	\item \textbf{Berechnung von Eingangs- und Ausgangswiderstand}
	\begin{equation}
	\begin{aligned}
	  R_{\text{in}} &= (\SI{100}{\kilo\ohm})||r_{\pi} = \SI{6297.7}{\ohm}\\
	  R_{\text{out}} &= (\SI{10}{\kilo\ohm})||r_o = \SI{9568.78}{\ohm}
	\end{aligned}
	\end{equation}
\end{enumerate}

\paragraph{(b)} Nun folgt die Repräsentation als Thévenin-Ersatzschaltung. Dazu legen wir die Ausgangspins offen und schauen, welcher Widerstand uns dort erwartet.
\[ R_{th} = r_o||\SI{10}{\kilo\ohm}||\SI{220}{\kilo\ohm} = \SI{9169.9}{\ohm} \]
Jetzt benötigen wir nur die an den offenen Pins anliegende Spannung:
\[ V_{th} = v_{be}g_mr_o||\SI{10}{\kilo\ohm} = -85.5v_i \]

\paragraph{(c)} Da wir in (b) bereits die Thévenin-Größen berechnet haben, ist Norton ein Selbstläufer:
\[ I_{th} = \frac{V_{th}}{R_{th}} = \frac{-85.5v_i}{\SI{9169.9}{\ohm}} \]

\subsubsection*{13.125}
Die Bias-Widerstände sind in dieser Aufgabe um den Faktor 10 angehoben worden – was die auftretenden Ströme um einen Faktor 10 verkleinert. Welche Auswirkungen hat das auf die Verstärkung und den Eingangs- und Ausgangswiderstand?
\begin{equation}
\begin{aligned}
  I_B  &= -\frac{\SI{0.7}{\volt} - \SI{5}{\volt}}{\SI{1}{\mega\ohm} + (1+\beta_f)(\SI{160}{\kilo\ohm})} = \SI{371.97}{\nano\ampere} \\
  I_C &= \beta_fI_B = (65)(\SI{371.97}{\nano\ampere}) = \SI{24.18}{\micro\ampere}\\
  V_{CE} &= \SI{10}{\volt} - I_C(\SI{100}{\kilo\ohm}) - I_E(\SI{160}{\kilo\ohm}) = \SI{3.65}{\volt} \\
  g_m &= \frac{I_C}{V_T} = \frac{\SI{24.18}{\micro\ampere}}{\SI{0.025}{\volt}} = \SI{0.967}{\milli\siemens} \\
  r_{\pi} &= \frac{\beta_0}{g_m} = \frac{65}{\SI{0.967}{\milli\siemens}} = \SI{67.2}{\kilo\ohm} \\
  r_o &= \frac{V_A + V_{CE}}{I_C} = \frac{\SI{53.65}{\volt}}{\SI{24.18}{\micro\ampere}} = \SI{2.22}{\mega\ohm} \\
  v_{be} &= v_i\frac{\SI{1}{\mega\ohm}||r_{\pi}}{\SI{330}{\ohm} + \SI{1}{\mega\ohm}||r_{\pi}} = 0.995v_i \\
  A_v^{CS} &= - \frac{v_{be}}{v_i}g_mr_o||\SI{100}{\kilo\ohm}||\SI{220}{\kilo\ohm} = -(0.995)(\SI{0.967}{\milli\siemens})(\SI{66.7}{\kilo\ohm}) = -64.18 \\
  R_{\text{in}} &= (\SI{1}{\mega\ohm})||r_{\pi} = \SI{62.97}{\kilo\ohm}\\
  R_{\text{out}} &= (\SI{100}{\kilo\ohm})||r_o = \SI{95.69}{\kilo\ohm}
\end{aligned}
\end{equation}

\subsection{Amplifier Power and Signal Range}
\subsubsection*{13.129}
In dieser Aufgabe soll der Energieverbrauch in jedem einzelnen Bauteil der Schaltung Fig. 13.40(a) berechnet werden bei $\beta_f = 75$.\\
Um die Leistung zu berechnen, müssen vorher alle Einzelströme berechnet werden. Das linke Steuernetzwerk wird dazu folgendermaßen vereinfacht:
\[ V_{th} = V_{CC}\frac{R_1}{R_1 + R_2} = \frac{96}{23}\,\text{V} \quad \quad R_{th} = \frac{R_1R_2}{R_1+R_2} = \SI{104.3}{\kilo\ohm} \]
Anschließend kann über die Maschengleichung der Basisstrom und damit auch Kollektor- und Emitterstrom berechnet werden:
\[ I_B = \frac{V_{th} - V_{BE}}{R_{th} + (\beta_f+1)R_E} = \SI{3.18}{\micro\ampere} \qquad I_C=\beta_f I_B = \SI{238.5}{\micro\ampere} \qquad I_E=(\beta_f + 1) I_B = \SI{241.7}{\micro\ampere} \]
Nun berechnen wir die Leistungsabgabe in den rechten Widerständen:
\[ P_{R_C} = I_C^2R_C = \SI{1.25}{\milli\watt} \qquad P_{R_E} = I_E^2R_E = \SI{0.76}{\milli\watt} \]
Um nun die fehlenden Leistungen in den anderen beiden Widerständen zu berechnen, scheint die Berechnung der Spannung an der Basis $V_B$ hilfreich zu sein:
\[ V_B = I_ER_E + V_{BE} = \SI{3.84}{\volt} \]
Es folgt nun also für die Widerstände $R_1$ und $R_2$:
\[ P_{R_1} = \frac{V_B^2}{R_1} = \SI{92.16}{\micro\watt} \qquad P_{R_2} = \frac{(V_{CC}-V_B)^2}{R_2} = \SI{0.222}{\milli\watt} \]
Als letztes folgt die Berechnung der Leistung in den Diodenübergängen:
\[ P_D = V_{CE}I_C + V_{BE}I_B = (\SI{3.61}{\volt})(\SI{238.5}{\micro\ampere}) + (\SI{0.7}{\volt})(\SI{3.18}{\micro\ampere}) = \SI{0.86}{\milli\watt} \]
In der Summe ergibt das eine Leistungsaufnahme von $P_S = \SI{3.18}{\milli\watt}$. Vergleicht man dazu die Gesamtleistung, welche Strom- und Spannungseinspeisung ergibt
\[ P_S^{\text{Sources}} = V_{CC}(I_C+I_2) = \SI{3.19}{\milli\watt}, \]
stellen wir eine gute Übereinkunft fest.

\subsubsection*{13.130}
Gleicher Aufgabentyp wie in (13.129) mit dem Unterschied dass es sich dieses Mal um einen Feldeffekttransistor handelt. Da hier kein $K_n$ gegeben ist, nehmen wir einfach die vorgegebenen Werte für den Arbeitspunkt: $(\SI{241}{\micro\ampere}, \SI{3.81}{\volt})$. Ebenfalls einfach zu bestimmen ist der zweite Teilstrom $I_2 = \frac{V_{DD}}{R_1+R_2} = \SI{3.24}{\micro\ampere}$. Nun können die Leistungen zügig berechnet werden:
\begin{equation}
\begin{aligned}
  P_{R_D} &= I_D^2R_D = \SI{1.28}{\milli\watt} \qquad &P_{R_S} &= I_D^2R_S = \SI{0.7}{\milli\watt} \\
  P_{R_1} &= \frac{V_G^2}{R_1} = \SI{15.78}{\micro\watt} \qquad &P_{R_2} &= \frac{(V_{CC}-V_G)^2}{R_2} = \SI{23.14}{\micro\watt} \\
  P_D &= V_{DS}I_D = \SI{0.92}{\milli\ampere} \qquad &P_S &= \SI{2.94}{\milli\watt} \\
  P_S^{\text{Sources}} &= V_{DD}(I_D+I_2) = \SI{2.93}{\milli\watt}
\end{aligned}
\end{equation}

\subsubsection*{13.131 - 13.135}
Aufgrund der Äquivalenz der Aufgaben, wird die Lösung hier nicht aufgeführt.

\subsubsection*{13.136}
Zuerst rufen wir uns die Kleinsignalbedingung für den Kollektorstrom ins Gedächtnis (Gleichung 13.41 Textbuch):
\[ |i_c| \leq 0.2I_C \]
Da $I_C = \frac{V_{CC}}{3R_C}$ gilt, können wir durch Substitution folgenden Ausdruck aufschreiben:
\[ i_c \leq 0.2 \frac{V_{CC}}{3R_C} \Leftrightarrow v_c \leq 0.2 \frac{V_{CC}}{3} = \frac{V_{CC}}{15} \]

\subsubsection*{13.137}
Zuerst schreiben beide Kleinsignalbedingungen auf (Gleichung 13.72 und 13.73):
\[ v_{gs} \leq 0.2(V_{GS} - V_{TN}) \qquad i_d \leq 0.4I_D \]
Den Arbeitsstrom können wir über $ I_D = \frac{K_n}{2}(V_{GS} - V_{TN})^2 = \SI{0.39}{\milli\ampere} $ berechnen. Das Limit für die Quelle beträgt
\[ |v_I| = |v_{gs}| \leq 0.2(0-(-\SI{1.25}{\volt})) = \SI{0.25}{\volt}. \]
Jetzt wird es knifflig. Wir erstellen für das Schaltbild einen AC-Ersatzschaltplan, vernachlässigen $ r_o $ und ersetzen $ g_mv_{gs} $ mit $ i_d $. Durch diese Substitutionen erhalten wir:
\[ v_{ds} = i_dR_D = 0.4(\SI{0.39}{\milli\ampere})(\SI{15}{\kilo\ohm}) = \SI{2.34}{\volt} \]
Zusätzlich muss für den Sättigungsbereich auch folgende Gleichung erfüllt sein:
\[ V_{DS} \geq v_{gs} - V_{TN} = \SI{1.5}{\volt} \]
Die Mindestversorgung soll mindestens
\[ V_{DD} \geq \SI{1.5}{\volt} + \SI{2.34}{\volt} + (\SI{0.39}{\milli\ampere})(\SI{15}{\kilo\ohm}) = \SI{9.69}{\volt} \] 
betragen.

\subsubsection*{13.138}

\end{document}