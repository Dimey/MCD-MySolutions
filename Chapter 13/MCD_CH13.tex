\documentclass[11pt,a4paper,titlepage]{article}
\usepackage[a4paper]{geometry}
\usepackage[utf8]{inputenc}
\usepackage[german]{babel}
\usepackage{lipsum}

\usepackage{amsmath, amssymb, amsfonts, amsthm, fouriernc, mathtools}
\usepackage{microtype} %improves the spacing between words and letters

\usepackage{graphicx}
\graphicspath{{./pics/}}

\usepackage{tabularx}
\usepackage{booktabs}
\usepackage{wasysym}
\usepackage{color, colortbl}


%%%%%%%%%%%%%%%%%%%%%%%%%%%%%%%%%%%%%%%%%%%%%%%%%%
%% COLOR DEFINITIONS
%%%%%%%%%%%%%%%%%%%%%%%%%%%%%%%%%%%%%%%%%%%%%%%%%%
\usepackage[svgnames]{xcolor} % Enabling mixing colors and color's call by 'svgnames'
%%%%%%%%%%%%%%%%%%%%%%%%%%%%%%%%%%%%%%%%%%%%%%%%%%
\definecolor{MyColor1}{rgb}{0.2,0.4,0.6} %mix personal color
\newcommand{\textb}{\color{Black} \usefont{OT1}{lmss}{m}{n}}
\newcommand{\blue}{\color{MyColor1} \usefont{OT1}{lmss}{m}{n}}
\newcommand{\blueb}{\color{MyColor1} \usefont{OT1}{lmss}{b}{n}}
\newcommand{\red}{\color{LightCoral} \usefont{OT1}{lmss}{m}{n}}
\newcommand{\green}{\color{Black} \usefont{OT1}{lmss}{m}{n}}
%%%%%%%%%%%%%%%%%%%%%%%%%%%%%%%%%%%%%%%%%%%%%%%%%%


%%%%%%%%%%%%%%%%%%%%%%%%%%%%%%%%%%%%%%%%%%%%%%%%%%
%% FONTS AND COLORS
%%%%%%%%%%%%%%%%%%%%%%%%%%%%%%%%%%%%%%%%%%%%%%%%%%
%    SECTIONS
%%%%%%%%%%%%%%%%%%%%%%%%%%%%%%%%%%%%%%%%%%%%%%%%%%
\usepackage{titlesec}
\usepackage{sectsty}
%%%%%%%%%%%%%%%%%%%%%%%%
%set section/subsections HEADINGS font and color
\sectionfont{\color{MyColor1}}  % sets colour of sections
\subsectionfont{\color{MyColor1}}  % sets colour of sections

%set section enumerator to arabic number (see footnotes markings alternatives)
\renewcommand\thesection{\arabic{section}} %define sections numbering
\renewcommand\thesubsection{\thesection.\arabic{subsection}} %subsec.num.

%define new section style
\newcommand{\mysection}{
	\titleformat{\section} [runin] {\usefont{OT1}{lmss}{b}{n}\color{MyColor1}} 
	{\thesection} {3pt} {} } 

%%%%%%%%%%%%%%%%%%%%%%%%%%%%%%%%%%%%%%%%%%%%%%%%%%
%		CAPTIONS
%%%%%%%%%%%%%%%%%%%%%%%%%%%%%%%%%%%%%%%%%%%%%%%%%%
\usepackage{caption}
\usepackage{subcaption}
%%%%%%%%%%%%%%%%%%%%%%%%
%\captionsetup[figure]{labelfont={color=Black}}

%%%%%%%%%%%%%%%%%%%%%%%%%%%%%%%%%%%%%%%%%%%%%%%%%%
%		!!!EQUATION (ARRAY) --> USING ALIGN INSTEAD
%%%%%%%%%%%%%%%%%%%%%%%%%%%%%%%%%%%%%%%%%%%%%%%%%%
%using amsmath package to redefine eq. numeration (1.1, 1.2, ...) 
%%%%%%%%%%%%%%%%%%%%%%%%
\renewcommand{\theequation}{\thesection.\arabic{equation}}

%set box background to grey in align environment 
\usepackage{etoolbox}% http://ctan.org/pkg/etoolbox
\makeatletter
\patchcmd{\@Aboxed}{\boxed{#1#2}}{\colorbox{black!15}{$#1#2$}}{}{}%
\patchcmd{\@boxed}{\boxed{#1#2}}{\colorbox{black!15}{$#1#2$}}{}{}%
\makeatother
%%%%%%%%%%%%%%%%%%%%%%%%%%%%%%%%%%%%%%%%%%%%%%%%%%


%%%%%%%%%%%%%%%%%%%%%%%%%%%%%%%%%%%%%%%%%%%%%%%%%%
%% DESIGN CIRCUITS
%%%%%%%%%%%%%%%%%%%%%%%%%%%%%%%%%%%%%%%%%%%%%%%%%%
\usepackage[siunitx, american, smartlabels, cute inductors, europeanvoltages]{circuitikz}
%%%%%%%%%%%%%%%%%%%%%%%%%%%%%%%%%%%%%%%%%%%%%%%%%%


\makeatletter
\let\reftagform@=\tagform@
\def\tagform@#1{\maketag@@@{(\ignorespaces\textcolor{red}{#1}\unskip\@@italiccorr)}}
\renewcommand{\eqref}[1]{\textup{\reftagform@{\ref{#1}}}}
\makeatother
\usepackage{hyperref}
\hypersetup{colorlinks=true}
\usepackage{hypcap}

%%%%%%%%%%%%%%%%%%%%%%%%%%%%%%%%%%%%%%%%%%%%%%%%%%
%% PREPARE TITLE
%%%%%%%%%%%%%%%%%%%%%%%%%%%%%%%%%%%%%%%%%%%%%%%%%%
\title{\blue MICROELECTRONIC CIRCUIT DESIGN \\	\blueb Solutions}
\author{Dimitri Haas}
\date{\today}
%%%%%%%%%%%%%%%%%%%%%%%%%%%%%%%%%%%%%%%%%%%%%%%%%%



\begin{document}
\maketitle
\setcounter{section}{12}
\section{Small-Signal Modeling and Linear Amplification}
\setcounter{subsection}{7}
\subsection{Small-Signal Models for Field-Effect Transistors}
\subsubsection*{13.80}
Folgende Werte sind bereits anhand der Aufgabe gegeben:
\begin{equation*}
\begin{aligned}
K_n &= \SI{300}{\micro\ampere\per\square\volt}\\
V_{TN} &= \SI{1}{\volt}\\
\lambda &= \SI{0.02}{\per\volt}
\end{aligned}
\end{equation*}
Ab welchem Drainstrom $I_D$ ist keine Spannungsverstärkung mehr vorhanden, also $\mu_f \leq 1$?\\
\\
Dazu suchen wir in Tabelle 13.3 nach der Formel für $\mu_f$. Diese lautet
\[\mu_f = \frac{1}{\lambda}\sqrt{\frac{2K_n}{I_D}}.\]
Wenn wir nun die Forderung einsetzen und die Gleichung nach $I_D$ umstellen, erhalten wir

\begin{equation*}
\begin{aligned}
 & &\frac{1}{\lambda}\sqrt{\frac{2K_n}{I_D}} &\leq 1 \\
& \Leftrightarrow & \frac{2K_n}{\lambda^2} &\leq I_D \\
& \Leftrightarrow & \frac{2(\SI{300}{\micro\ampere\per\square\volt})}{(\SI{0.02}{\per\volt})^2} &\leq I_D \\
& \Leftrightarrow & \SI{1.5}{\ampere} &\leq I_D
\end{aligned}
\end{equation*}\\
Damit liegt ab einem Drainstrom von $I_D = \SI{1.5}{\ampere}$ keine Spannungsverstärkung mehr vor.

\subsubsection*{13.81}
In dieser Aufgabe soll überprüft werden, ob die Ausdruck
\[\frac{2v_{gs}}{V_{GS}-V_{TN}}\] 
eine geeignete lineare Approximation für Ausdruck
\[\left(1 + \frac{v_{gs}}{V_{GS}-V_{TN}}\right)^2-1\]
darstellt, wenn $v_{gs} = 0.2(V_{GS}-V_{TN})$ gilt.\\
\\
Dazu setzen wir im ersten Schritt in die Ausgangsgleichung das vorgegeben $v_{gs}$ ein und erhalten
\[\left(1 + \frac{0.2(V_{GS}-V_{TN})}{V_{GS}-V_{TN}}\right)^2-1 = (1 + 0.2)^2-1 = 0.44\]
Das gleiche wiederholen wir nun für angenäherten Ausdruck:
\[\frac{(2)(0.2)(V_{GS}-V_{TN})}{V_{GS}-V_{TN}} = 0.4\]
Betrachten wir nun den relativen Fehler
\[f = \frac{0.4 - 0.44}{0.44}\SI{100}{\percent} = \SI{-9.1}{\percent} \approx \SI{-10}{\percent}\]
sehen wir, dass der Fehler bei etwa $\SI{10}{\percent}$ liegt und damit eine gute Näherung darstellt.\\
\\
Führen wir die gleichen Berechnungen für $v_{gs} = 0.4(V_{GS}-V_{TN})$ durch, ergibt sich ein Fehler von
\[f = \frac{0.8 - 0.96}{0.96}\SI{100}{\percent} = \SI{-16.7}{\percent} \approx \SI{-20}{\percent}.\]
Man sieht also, dass das Signal wirklich sehr klein sein muss, um mithilfe der obigen Annäherung ein gutes Ergebnis zu erzielen.

\subsubsection*{13.84}
Gegeben ist ein Common-Source-Amplifier mit folgenden Spezifikationen:

\begin{equation*}
\begin{aligned}
R_\text{out}^{CS} &= \SI{100}{\kilo\ohm}\\
\lambda &= \SI{0.02}{\per\volt}\\
V_{DD} &= \SI{15}{\volt}
\end{aligned}
\end{equation*}
Da ein CS-Amplifier so gestaltet wird, dass etwa \SI{50}{\percent} der Versorgungsspannung über dem Drain-Widerstand abfallen und die Ausgangsspannung weitestgehend der Spannung über dem Drain-Widerstand entspricht, können wir einen Teil des Arbeitspunktes bereits vorzeitig angeben:
\[V_{DS} = \frac{V_{DD}}{2} = \frac{\SI{15}{\volt}}{2} = \SI{7.5}{\volt}\]
Nun gilt es noch einen fehlenden Transistor-Parameter zu bestimmen, welcher vollkommen unabhängig von der äußeren CS-Beschaltung ist:
\begin{equation}\label{eq:r_0} 
r_0 = \frac{\frac{1}{\lambda}+V_{DS}}{I_D}
\end{equation}
Allerdings enthält Gleichung (\ref{eq:r_0}) noch zwei Unbekannte. Wir müssen deshalb noch einen weiteren Zusammenhang ausfindig machen. Fündig wird man, wenn man sich klarmacht, dass der Ausgangswiderstand eine Parallelanordnung der Widerstände $r_0$ und $R_D$ darstellt. Da wir in der Aufgabenstellung seinen Wert übermittelt bekommen haben, können wir schreiben:
\begin{equation}\label{eq:r_out} 
R_\text{out} = R_{\text{out}}^{CS} = \frac{r_0 R_D}{r_0+R_D}
\end{equation}
Setzt man nun Gleichung (\ref{eq:r_0}) in (\ref{eq:r_out}) ein und verwendet die Beziehung $R_D=\frac{V_{DS}}{I_D}$, erhält man:
\begin{equation} \label{eq:r_outAndId}
\begin{aligned}
R_\text{out}^{CS} &= \frac{\frac{\frac{1}{\lambda}+V_{DS}}{I_D}\frac{V_{DS}}{I_D}}{\frac{\frac{1}{\lambda}+V_{DS}}{I_D}+\frac{V_{DS}}{I_D}} \\
&= \frac{1}{I_D}\frac{\frac{1}{\lambda}+V_{DS}}{\frac{1}{\lambda V_{DS}}+2} \\
&\approx \frac{6.63}{I_D}
\end{aligned}
\end{equation}
Da uns $R_\text{out}^{CS}$ bereits in der Aufgabenstellung gegeben wurde, können wir mithilfe von Gleichung (\ref{eq:r_outAndId}) den fehlenden Arbeitspunkt-Parameter bestimmen:
\begin{equation}
I_D = \frac{6.63}{R_\text{out}^{CS}} = \frac{6.63}{\SI{100}{\kilo\ohm}} = \SI{66.3}{\micro\ampere}
\end{equation}
Der Arbeitspunkt lautet somit abschließend $(\SI{7.5}{\volt},\SI{66.3}{\micro\ampere})$. Diese Aufgabe kann man sich mithilfe von Figure 13.18 im Buch auf Seite 793 veranschaulichen. Hier müssen lediglich die BJT- durch MOSFET-Bauteile ersetzt werden.

\subsubsection*{13.85}
In einem CS-Verstärker, ist der Arbeitspunkt gesucht, bei dem wir einen Eingangswiderstand von $R_{\text{in}}^{CS} = \SI{2}{\mega\ohm}$ erhalten. Der darin verbaute Transistor besitzt folgende Parameter:
\begin{equation*}
\begin{aligned}
K_n &= \SI{500}{\micro\ampere\per\square\volt}\\
V_{TN} &= \SI{1}{\volt}\\
\lambda &= \SI{0.02}{\per\volt}
\end{aligned}
\end{equation*}
Gearbeitet wird mit einer Versorgungsspannung von \SI{12}{\volt}.\\
\\
Die obige Aufgabe kann auch ohne Studium des Abschnitts 13.10.5 im Buch gelöst werden: Da der Transistor selbst einen unendlichen hohen Eingangswiderstand hat, bleibt nur noch $R_G$ als Widerstand übrig. Somit ist der Eingangwiderstand nur von den beiden Gate-Widerständen nicht aber vom Arbeitspunkt abhängig. Jeder Arbeitspunkt ermöglicht damit den oben vorgegeben Eingangswiderstand.

\end{document}