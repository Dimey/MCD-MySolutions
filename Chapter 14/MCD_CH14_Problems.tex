\documentclass[11pt,a4paper,titlepage]{article}
\usepackage[a4paper]{geometry}
\usepackage[utf8]{inputenc}
\usepackage[german]{babel}
\usepackage{lipsum}

\usepackage{amsmath, amssymb, amsfonts, amsthm, fouriernc, mathtools}
\usepackage{microtype} %improves the spacing between words and letters

\usepackage{graphicx}
\graphicspath{{./pics/}}

\usepackage{tabularx}
\usepackage{booktabs}
\usepackage{wasysym}
\usepackage{color, colortbl}


%%%%%%%%%%%%%%%%%%%%%%%%%%%%%%%%%%%%%%%%%%%%%%%%%%
%% COLOR DEFINITIONS
%%%%%%%%%%%%%%%%%%%%%%%%%%%%%%%%%%%%%%%%%%%%%%%%%%
\usepackage[svgnames]{xcolor} % Enabling mixing colors and color's call by 'svgnames'
%%%%%%%%%%%%%%%%%%%%%%%%%%%%%%%%%%%%%%%%%%%%%%%%%%
\definecolor{MyColor1}{rgb}{0.65,0.165,0.165} %mix personal color
\newcommand{\textb}{\color{Black} \usefont{OT1}{lmss}{m}{n}}
\newcommand{\blue}{\color{MyColor1} \usefont{OT1}{lmss}{m}{n}}
\newcommand{\blueb}{\color{MyColor1} \usefont{OT1}{lmss}{b}{n}}
\newcommand{\red}{\color{LightCoral} \usefont{OT1}{lmss}{m}{n}}
\newcommand{\green}{\color{Black} \usefont{OT1}{lmss}{m}{n}}
%%%%%%%%%%%%%%%%%%%%%%%%%%%%%%%%%%%%%%%%%%%%%%%%%%


%%%%%%%%%%%%%%%%%%%%%%%%%%%%%%%%%%%%%%%%%%%%%%%%%%
%% FONTS AND COLORS
%%%%%%%%%%%%%%%%%%%%%%%%%%%%%%%%%%%%%%%%%%%%%%%%%%
%    SECTIONS
%%%%%%%%%%%%%%%%%%%%%%%%%%%%%%%%%%%%%%%%%%%%%%%%%%
\usepackage{titlesec}
\usepackage{sectsty}
%%%%%%%%%%%%%%%%%%%%%%%%
%set section/subsections HEADINGS font and color
\sectionfont{\color{MyColor1}}  % sets colour of sections
\subsectionfont{\color{MyColor1}}  % sets colour of sections

%set section enumerator to arabic number (see footnotes markings alternatives)
\renewcommand\thesection{\arabic{section}} %define sections numbering
\renewcommand\thesubsection{\thesection.\arabic{subsection}} %subsec.num.

%define new section style
\newcommand{\mysection}{
	\titleformat{\section} [runin] {\usefont{OT1}{lmss}{b}{n}\color{MyColor1}} 
	{\thesection} {3pt} {} } 

%%%%%%%%%%%%%%%%%%%%%%%%%%%%%%%%%%%%%%%%%%%%%%%%%%
%		CAPTIONS
%%%%%%%%%%%%%%%%%%%%%%%%%%%%%%%%%%%%%%%%%%%%%%%%%%
\usepackage{caption}
\usepackage{subcaption}
%%%%%%%%%%%%%%%%%%%%%%%%
%\captionsetup[figure]{labelfont={color=Black}}

%%%%%%%%%%%%%%%%%%%%%%%%%%%%%%%%%%%%%%%%%%%%%%%%%%
%		!!!EQUATION (ARRAY) --> USING ALIGN INSTEAD
%%%%%%%%%%%%%%%%%%%%%%%%%%%%%%%%%%%%%%%%%%%%%%%%%%
%using amsmath package to redefine eq. numeration (1.1, 1.2, ...) 
%%%%%%%%%%%%%%%%%%%%%%%%
\renewcommand{\theequation}{\thesection.\arabic{equation}}

%set box background to grey in align environment 
\usepackage{etoolbox}% http://ctan.org/pkg/etoolbox
\makeatletter
\patchcmd{\@Aboxed}{\boxed{#1#2}}{\colorbox{black!15}{$#1#2$}}{}{}%
\patchcmd{\@boxed}{\boxed{#1#2}}{\colorbox{black!15}{$#1#2$}}{}{}%
\makeatother
%%%%%%%%%%%%%%%%%%%%%%%%%%%%%%%%%%%%%%%%%%%%%%%%%%


%%%%%%%%%%%%%%%%%%%%%%%%%%%%%%%%%%%%%%%%%%%%%%%%%%
%% DESIGN CIRCUITS
%%%%%%%%%%%%%%%%%%%%%%%%%%%%%%%%%%%%%%%%%%%%%%%%%%
\usepackage[siunitx, american, smartlabels, cute inductors, europeanvoltages]{circuitikz}
%%%%%%%%%%%%%%%%%%%%%%%%%%%%%%%%%%%%%%%%%%%%%%%%%%


\makeatletter
\let\reftagform@=\tagform@
\def\tagform@#1{\maketag@@@{(\ignorespaces\textcolor{red}{#1}\unskip\@@italiccorr)}}
\renewcommand{\eqref}[1]{\textup{\reftagform@{\ref{#1}}}}
\makeatother

%%%%%%%%%%%%%%%%%%%%%%%%%%%%%%%%%%%%%%%%%%%%%%%%%%
%% PREPARE TITLE
%%%%%%%%%%%%%%%%%%%%%%%%%%%%%%%%%%%%%%%%%%%%%%%%%%
\title{\blue MICROELECTRONIC CIRCUIT DESIGN \\	\blueb Chapter 14 \\ Single-Transistor Amplifiers \\ Solution Manual}
\author{Dimitri Haas}
\date{\today}
%%%%%%%%%%%%%%%%%%%%%%%%%%%%%%%%%%%%%%%%%%%%%%%%%%



\begin{document}
\maketitle
\setcounter{section}{14}
\subsection{Amplifier Classification}
\subsubsection*{14.1}
Zeichne das AC-Ersatzschaltbild und ordne folgenden Konfigurationen zu: C-S, C-G, C-D, C-E, C-B, C-C und unnütz.\\
\textit{Die Zeichnungen erfolgen zunächst auf Papier mit Stift und werden alsbald nachgetragen mithilfe von Circuitz.}
\paragraph{(a)} Common-Collector. Eingang an der Base und Ausgang am Emitter.
\paragraph{(b)} Common-Collector. Eingang an der Base und Ausgang am Emitter.
\paragraph{(c)} Common-Collector. Eingang an der Base und Ausgang am Emitter.
\paragraph{(d)} Common-Collector. Eingang an der Base und Ausgang am Emitter.
\paragraph{(e)} Common-Collector. Eingang an der Base und Ausgang am Emitter.
\paragraph{(f)} Common-Collector. Eingang an der Base und Ausgang am Emitter.
\paragraph{(g)} Common-Collector. Eingang an der Base und Ausgang am Emitter.
\paragraph{(h)} Common-Collector. Eingang an der Base und Ausgang am Emitter.
\paragraph{(i)} Common-Collector. Eingang an der Base und Ausgang am Emitter.
\paragraph{(j)} Common-Collector. Eingang an der Base und Ausgang am Emitter.
\paragraph{(k)} Common-Collector. Eingang an der Base und Ausgang am Emitter.
\paragraph{(l)} Common-Collector. Eingang an der Base und Ausgang am Emitter.
\paragraph{(m)} Common-Collector. Eingang an der Base und Ausgang am Emitter.
\paragraph{(n)} Common-Collector. Eingang an der Base und Ausgang am Emitter.
\paragraph{(o)} Common-Collector. Eingang an der Base und Ausgang am Emitter.
\paragraph{(p)} Common-Collector. Eingang an der Base und Ausgang am Emitter.
\paragraph{(q)} Common-Collector. Eingang an der Base und Ausgang am Emitter.

\subsubsection*{14.2}

\subsubsection*{14.3}

\subsubsection*{14.4}

\subsubsection*{14.5}

\subsubsection*{14.6}

\subsubsection*{14.7}

\subsubsection*{14.8}

\subsubsection*{14.9}

\subsubsection*{14.10}

\subsubsection*{14.11}

\subsubsection*{14.12}

\subsubsection*{14.13}

\subsubsection*{14.14}

\subsubsection*{14.15}

\subsection{Inverting Amplifiers – Common-Emitter/Common-Source Circuits}

\subsubsection*{14.16}

\subsubsection*{14.17}

\subsubsection*{14.18}

\subsubsection*{14.19}

\subsubsection*{14.20}

\subsubsection*{14.21}

\subsubsection*{14.22}

\subsubsection*{14.23}

\subsubsection*{14.24}

\subsubsection*{14.25}

\subsubsection*{14.26}

\subsubsection*{14.27}

\subsubsection*{14.28}

\subsection{Follower Circuits – Common-Collector/Common-Drain Amplifiers}

\subsubsection*{14.29}

\subsubsection*{14.30}

\subsubsection*{14.31}

\subsubsection*{14.32}

\subsubsection*{14.33}

\subsubsection*{14.34}

\subsubsection*{14.35}

\subsubsection*{14.36}

\subsubsection*{14.37}

\subsubsection*{14.38}

\end{document}